\chapter{Design}\label{chapter:design} \

\section{Section} \

1.	Explain why decoherent noise is used and not coherent. \

  a. Coherent noise will probably just shift the bias. \

  b. Coherent noise might add too much noise (quadratic growth). \

  c. True random noise is required to improve generalization. \

% TODO: Do an experiment implementing coherent noise to prove this claim.
% Use https://pennylane.ai/qml/demos/tutorial_variational_classifier/ - circuit-centric quantum classifier ansatz

- introduce reproduction of variational circuits inspired by circuit centric ansatz. \

- describe that state preparation influences a lot for coherent noise. \

\subsection{Datasets}\label{subsection:datasets} \

% TODO: Describe the datasets that we are using and why.
% TODO: Maybe perform a small feature engineering analysis on each dataset.

\subsection{Variational Quantum Classifiers Training}\label{subsection:vqc_training} \

% TODO: State the result of training QVC with regards to the chosen datasets.
% TODO: Compare to paper adn state why they might be valid results.

\subsection{State Preparation with Coherent Noise}\label{subsection:state_preparation_noise} \

In the previous subsection the trained variational circuit classifiers
utilized the amplitude embedding technique to encode classical data
into the \ac{qml} model. Pennylanes's \colorbox{inline_gray}{\lstinline|AmplitudeEmbedding|}
was utilized to encode the data accordingly. In this subsection we will
present the impact of state preparation with coherent noise on quantum
circuits. \

In order to simulate coherent noise we used Qiskit's Aer noise simulator.
Qiskit's noise models can be linked with Pennylane by declaring them as
an argument when creating a device. We attach coherent noise per gate
with the noise's corresponding unitary matrix. The unitary matrices are
derived with their respective rotational matrices \(R_{Y}\), \(R_{Z}\) and 
\(CR_{X}\) with a small \(\theta = \epsilon\). In this case we set 
\(\epsilon\) to be around 0.175 to simulate a \(10^{\circ}\) gate
miscalibration. \

The test circuits are composed first by an
\colorbox{inline_gray}{\lstinline|AmplitudeEmbedding|} state preparation
layer. Then, a single quantum gate follows that will characterize the
test circuit. In total there are three test circuits that describe the
effect of coherent noise on the \(R_{Y}\), \(R_{Z}\) and \ac{cnot} gates.
These gates were chosen because they are utilized by the strongly
entangling layers ansatz. For the rotational gates, a \(\theta = \pi\)
angle was chosen to simulate their equivalent non-rotational gates.\

In Table~\ref{tab:ry_ideal} we can observe the ideal noiseless results
for the \(R_{Y}(\pi)\) gate. This table yields the same result as applying
the Pauli Y gate. We use the Pauli Z matrix as the observable. The 
projective measurement values will be between the range of -1 and 1.

\begin{table}[h]
  \centering
  \begin{tabular}{|c|c|c|}
    \hline
    Quantum State & \(R_{Y}\left(\pi\right)\) & \(\expval{Z}\) \\
    \hline
    \(\ket{0}\) & \(i\ket{1}\) & -1 \\
    \hline
    \(\ket{1}\) & \(-i\ket{0}\) & 1 \\
    \hline
    \(\ket{+}\) & \(-i\ket{-}\) & 0 \\
    \hline
  \end{tabular}
  \caption{Expectation value with Pauli Z observable of \(R_{Y}\) gate with \(\theta = \pi\).}\label{tab:ry_ideal}
\end{table} \

The calculated expectation values for the \(R_{Y}(\theta + \epsilon)\) gate
can be found in Table~\ref{tab:ry_iso_noise}. We can observe that the
measurements from the resulting quantum state are perturbed. For the
computational basis states the expected value is equally reduced in magnitude.
For the superposition state \(\ket{+}\), the expected value 0 is shifted
by the coherent noise. The magnitude of the disturbances from the
measurements is directly proportional to \(\epsilon\). \

\begin{table}[h]
  \centering
  \begin{tabular}{|c|c|}
    \hline
    Quantum State & \(\expval{Z}\) for \(R_{Y}\left(\pi+\epsilon\right)\) \\
    \hline
    \(\ket{0}\) & -0.985 \\
    \hline
    \(\ket{1}\) & 0.985 \\
    \hline
    \(\ket{+}\) &  0.174 \\
    \hline
  \end{tabular}
  \caption{Expectation value with Pauli Z observable of \(R_{Y}\) gate with \(\theta = \pi\) and \(\epsilon = 10^{\circ}\).}\label{tab:ry_iso_noise}
\end{table} \

The experimental results from adding coherent noise to the
\(R_{Y}(\theta)\) gate are presented in Table~\ref{tab:ry_real_noise}. 
While for the \(\ket{0}\) state the result does match the calculated
-0.985 value, for the other two tested quantum states there is a
notable difference between the calculated and the experimental results. \

\begin{table}[h]
  \centering
  \begin{tabular}{|c|c|}
    \hline
    Quantum State & \(\expval{Z}\) for \(R_{Y}\left(\pi+\epsilon\right)\) \\
    \hline
    \(\ket{0}\) & -0.985 \\
    \hline
    \(\ket{1}\) & 0.940 \\
    \hline
    \(\ket{+}\) &  0.340 \\
    \hline
  \end{tabular}
  \caption{Obtained expectation value with Pauli Z observable of \(R_{Y}\) gate with \(\theta = \pi\) and \(\epsilon = 10^{\circ}\).}\label{tab:ry_real_noise}
\end{table} \

In relation to the test from the \(R_{Z}(\theta + \epsilon)\) gate, we
are not able to observe any effect of either the rotation or the niose.
This is because any Z transformation only modifies the phase of the
quantum state, which cannot be measured with the Pauli Z observable.
However, it is important to note that in the experimental results
a difference of around 0.001 was observed in the result for the \(\ket{+}\)
state measurement. \

For the \ac{cnot} gate circuit we tested both cases (a control qubit on the
first wire and a target qubit for the second wire and vice versa). For
the first case of the \ac{cnot}  gate circuit test, the noiseless expectation
values with the Pauli Z observable can be found in Table
~\ref{tab:cnot_ideal}. Similarly to the \(R_{Y}(\theta)\) circuit test,
the expectation values lie between a range of -1 and 1. Contrarily,
for the case of the \ac{cnot}  gate we measure at the end of the circuit the
two qubits from the quantum state. \

\begin{table}[h]
  \centering
  \begin{tabular}{|c|c|c|}
    \hline
    Quantum State & \ac{cnot} & \(\expval{Z}\) \\
    \hline
    \(\ket{00}\) & \(\ket{00}\) & [1, 1] \\
    \hline
    \(\ket{01}\) & \(\ket{01}\) & [1, -1] \\
    \hline
    \(\ket{10}\) & \(\ket{11}\) & [-1, -1] \\
    \hline
    \(\ket{11}\) & \(\ket{10}\) & [-1, 1] \\
    \hline
  \end{tabular}
  \caption{Expectation value with Pauli Z observable of \ac{cnot} gate.}\label{tab:cnot_ideal}
\end{table} \

In Table~\ref{tab:cnot_iso_noise} we can find the calculated
projective measurement with the Pauli Z matrix for the \ac{cnot} gate.
First of all we can observe that for the quantum states \(\ket{00}\)
and \(\ket{01}\) the values don't incur in any noise as the control
qubit is set to 0 and the Pauli X rotation is not performed on the target
qubit. Notwithstanding, for \(\ket{10}\) and \(\ket{11}\), where the
control qubit is set, noise should be present in the second qubit's
measurements with an equal magnitude. \

\begin{table}[h]
  \centering
  \begin{tabular}{|c|c|c|}
    \hline
    Quantum State & \ac{cnot} & \(\expval{Z}\) \\
    \hline
    \(\ket{00}\) & \(\ket{00}\) & [1, 1] \\
    \hline
    \(\ket{01}\) & \(\ket{01}\) & [1, -1] \\
    \hline
    \(\ket{10}\) & \(\ket{11}\) & [-1, -0.985] \\
    \hline
    \(\ket{11}\) & \(\ket{10}\) & [-1, 0.985] \\
    \hline
  \end{tabular}
  \caption{Expectation value with Pauli Z observable of \ac{cnot} gate with \(\epsilon = 10^{\circ}\).}\label{tab:cnot_iso_noise}
\end{table} \

The experimental results from adding coherent noise to the
\ac{cnot} gate are presented in Table~\ref{tab:cnot_real_noise}.
Although the result for the \(\ket{00}\) state is the same as the
previously calculated value, the experimental values obtained for
the other three states differ. For the \(\ket{01}\) state there
should be no noise present, however, both qubit measurements
show disturbances in their value. Regarding the \(\ket{10}\) state
the measurement values for both qubits are inverted. Finally,
for the \(\ket{11}\) state the first qubit shows signs of noise
while it should not, while the second qubit exhibits a different noise
magnitude. \

\begin{table}[h]
  \centering
  \begin{tabular}{|c|c|c|}
    \hline
    Quantum State & \ac{cnot} & \(\expval{Z}\) \\
    \hline
    \(\ket{00}\) & \(\ket{00}\) & [1, 1] \\
    \hline
    \(\ket{01}\) & \(\ket{01}\) & [0.992, -0.996] \\
    \hline
    \(\ket{10}\) & \(\ket{11}\) & [-0.985, -1] \\
    \hline
    \(\ket{11}\) & \(\ket{10}\) & [-0.992, 0.996] \\
    \hline
  \end{tabular}
  \caption{Obtained expectation value with Pauli Z observable of \ac{cnot} gate with \(\epsilon = 10^{\circ}\).}\label{tab:cnot_real_noise}
\end{table} \

% TODO: coherent noise didn't stay unitarily consistent aka it had some variation which is also not expected

% TODO: state preparation with different lengths.

