\chapter{Introduction}\label{chapter:introduction}

- Quantum mechanics explained, jump to quantum computing. (Quantum Computing: Progress and Prospects)~\cite{national_academies_of_sciences_engineering_and_medicine_quantum_2019}
- Quantum algorithms are better for specific tasks. (Quantum Advantage papers)~\cite{shor_polynomial-time_1997, van_dam_quantum_2006, hallgren_polynomial-time_2007}
- Shor's algorithm breaks current asymmetric cryptographic implementations. (Shor)~\cite{shor_polynomial-time_1997}

\section{Motivation}
- These quantum algorithms can't be run due to noise. (NISQ)~\cite{preskill_quantum_2018}
- Error mitigation and fault tolerance (Quantum computing)~\cite{shor_quantum_nodate}, improving hw, is important and still being researched. (NISQ)~\cite{preskill_quantum_2018}
- Intro to machine learning, its uses and widespread popularity. Presents possible risks and challenges. (On the opportunities and risks of foundational models)~\cite{bommasani_opportunities_2022}
- What is QML and why are people doing it? (Machine Learning with Quantum Computers)~\cite{schuld_machine_2021}

- Adversarial attacks can be crafted to force a ML model to missclassify an input. (Intriguing properties of NN)~\cite{szegedy_intriguing_2014}
- Adversarial attacks can be transferred in between models. (Transferability in Machine Learning: from Phenomena to Black-Box Attacks using Adversarial Samples)~\cite{papernot_transferability_2016}
- Adversarial training as a defense for adversarial attacks enhancing generalization. (Explaining and harnessing | intriguing properties of NN)~\cite{goodfellow_explaining_2015, szegedy_intriguing_2014}
- Present the limitations of adversarial training in large-scale systems. (AML at scale)~\cite{kurakin_adversarial_2017}
- Describe the potential uses of noise in QML as a potential increase in robustness. (QML a classical perspective)~\cite{ciliberto_quantum_2018}

TODO\@:
- Could we claim that noise in QML might also be a defense against data poisoning attacks? (AML at scale)~\cite{kurakin_adversarial_2017}

\section{Research Goals}
1. Test the effect of different types of noise in QML regarding robustness.
2. Test the effect of noise in different parts of the QML circuits.
3. Test the effect of noise between VQA and Kernel methods.
4. Test the models with different types of adversarial attacks (FGSM, CaW, PGD)

\section{Outline}
Write here what is the general structure of the thesis.