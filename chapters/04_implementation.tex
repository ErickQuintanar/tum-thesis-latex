\chapter{Implementation}\label{chapter:implementation} \

In Chapter~\ref{chapter:implementation} we will introduce
the methods used to implement the experiments to answer
the research goals. In Section~\ref{section:vqa_training}
we will present how the \ac{qml} models are trained.
Futhermore, in Section~\ref{section:vqa_attacks} the
attack methodology on the \ac{vqa} model will be described. \

\section{Variational Quantum Algorithm Model Training}\label{section:vqa_training} \

In this section we will specify how the \ac{vqa} models
were trained. In Subsection~\ref{subsection:preprocess},
we will describe the dataset preprocessing methodology required
for the training. Later on in Subsection~\ref{subsection:config},
the configuration files utilized by the training pipeline will
be defined. Afterwards, in Subsection~\ref{subsection:noise_injection}
we will present how the noise is injected to the
\ac{qml} models during training and evaluation.
Finally, the pipeline required for the training will
be presented in Subsection~\ref{subsection:pipeline}. \

\subsection{Datasets Preprocessing}\label{subsection:preprocess} \

In order to reduce the execution time of the model training, all the
datasets mentioned in the Section~\ref{section:datasets} were
stored preprocessed. All the datasets but the Plus-Minus dataset
were retrieved using OpenML~\cite{vanschoren_openml_2014}. We
obtained the Plus-Minus dataset by asking for it to the authors of 
~\cite{wendlinger_comparative_2024}. \

The preprocessing methodology encompasses for all the datasets the
removal of data points that are missing at least one feature.
Then, duplicate data points are removed to avoid overemphasizing
these elements and creating a bias in the data distribution. In
case the target labels are non-numeric, we modified them to
represent numerical classes. Finally, all the datasets except
the Iris Flower dataset are normalized using the standard
deviation and rescaled between a range of \(0\) and \(1\). This
is done with Scikit-learn~\cite{pedregosa_scikit-learn_2011}
and its purpose is to prevent feature dominance and to quicken
training convergence. \

There are two last details with regards to the preprocessing
of the datasets. Firstly, as mentioned in Subsection
~\ref{subsection:mnist}, the images from MNIST were downsampled
to be able to process them with a \ac{vqa}. Secondly, the majority
class from the \ac{pid} dataset was reduced to obtain a balanced
dataset with regards to the class distribution. \

\subsection{Configuration File}\label{subsection:config} \

The configuration files are essential to the training
pipeline. Its contents define the dataset to be used,
which type of \ac{qml} model is going to be trained,
which noise model will be injected, and some hyperparameters
regarding the \ac{qml} model's architecture. These hyperparameters
are the number of qubits, the number of layers, the number of
classes, the learning rate, the batch size, and the epochs. \

\begin{lstlisting}[language=Python, caption={Example configuration file.}, label=lst:config_file]
  {
    "dataset" : "diabetes",
    "qml_model" : "pqc",
    "noise_model" : "none",
    "num_qubits" : 3,
    "num_layers" : 40,
    "num_classes" : 2,
    "learning_rate" : 0.0005,
    "batch_size" : 16,
    "epochs" : 10
  }
\end{lstlisting}

An example configuration file can be seen in Listing
~\ref{lst:config_file}. This configuration file is given
to the training pipeline. Once the training is finished,
two more fields will be added to the configuration file.
These fields are the model accuracy on the test set and an
\(id\) to help identify the resulting model weights. \

\subsection{Noise Injection}\label{subsection:noise_injection} \

\subsection{Training Pipeline}\label{subsection:pipeline} \

cumbersome to create manually the config files, so script does the job
with specific noise models and required arguments for them.

\section{Adversarial Attacks on Variational Quantum Algorithm Model}\label{section:vqa_attacks} \

Mention cleverhans library and how it is used. Mention pipeline to create the adversarial samples. Mention the attack strenght ratio, take into account that the samples are being produced over preprocessed data.

\section{Experiments}\label{section:experiments} \

Possible experiments: \

a. Change the type of encoding \

b. Variational quantum circuits and kernels \

 1. Amplitude damping (open quantum systems / how the quantum system is affected by its environment) \

 2. Phase damping (loss of information but not energy) \

 3. Simpler phase and bit flips \