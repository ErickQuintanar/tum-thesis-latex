\chapter{Theoretical Background}\label{chapter:background} \

In Chapter \ref{chapter:background} we will introduce the
background information that is required to understand
the main ideas of this thesis. First we introduce
the basic concepts of quantum computing. Then we will
describe advanced concepts regarding quantum noise. We
assume some baseline \ac{ml} knowledge, however, we
will provide an outlook into \ac{qml}. Finally, we
present several types of adversarial machine learning
and adversarial training as a defense mechanism. \

\section{Fundamentals of Quantum Computing}
- Introduce the qubit~\cite{schumacher}, 8 \
\begin{equation}\label{eq:qubit}
    \ket{0} = \begin{pmatrix}
                1 \\ 0
              \end{pmatrix} \qquad \qquad
    \ket{1} = \begin{pmatrix}
                0 \\ 1
              \end{pmatrix}
\end{equation}
    - Introuce Bloch sphere \
    - Introduce superposition \
    - Introduce many qubits system \
    - Introduce entanglement, 11 \ 
- Introduce quantum gates \
    - Equate a gate with a unitary matrix \
    - Pauli Gates \
    - Hadamard Gates \
- Introduce quantum circuits \
- Introduce quantum measurement \
- Introduce the density operator \
- Introduce quantum algorithms* \

i.	Introduce the required quantum concepts for the reader to understand noise in quantum computing.

\section{Quantum Noise}
i.	Describe the types of noise that can occur.
ii.	Explain where can noise occur.
iii.	State how noise can be simulated.

\section{Quantum Machine Learning}
i.	Present the difference between QML and classical ML\@.
ii.	Introduce variational quantum circuits.
iii.	Explain quantum kernel methods.

\section{Adversarial Machine Learning}
i.	State generalization problems.
ii.	Present different attacks such as FGSM, C\&W, and PGD\@.
iii.	Introduce adversarial training as defence mechanism against adversarial attacks.
iv.	Explain the relationship between general accuracy and adversarial resilience.