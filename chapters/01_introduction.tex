\chapter{Introduction}\label{chapter:introduction} \

In recent years the interest on techniques to utilize quantum mechanics has been rising.
One of the many applications is quantum computing, where devices based on the laws
of quantum theory are exploited to process information~\cite{national_academies_of_sciences_engineering_and_medicine_quantum_2019}.
Although current classical computers have become very powerful, they still struggle to
process many applications that quantum computers can in theory easily solve. \

Some quantum algorithms for quantum computers have been proposed that highly
outperform a classical computer with the best known algorithms~\cite{shor_polynomial-time_1997, van_dam_quantum_2006, hallgren_polynomial-time_2007}.
These quantum algorithms solve in polynomial time problems that quickly become
intractable to solve in a classical computer, as they normally grow exponentially.
The most famous algorithm is Shor's algorithm~\cite{shor_polynomial-time_1997}, which can find the prime factors
of an integer. It is of special interest because if quantum devices
were able to execute it, the current confidentiality and integrity guarantees
that the RSA~\cite{rivest_method_1978} cryptographic mechanism offers would be violated. \

\section{Motivation} \

Even though the previously mentioned quantum algorithms would surpass the performance
of the best classical ones, they still can not be executed on current quantum
computers due to noise~\cite{preskill_quantum_2018} and not having enough logical
qubits to perform the operations. This noise occurs because
current quantum devices are not completely isolated from the environment and every
time we perform an operation on them we introduce a disturbance. This type of device
is known as \ac{nisq}, meaning that there will be significant noise when operating
the quantum device. \

In order to reduce the influence of noise in \ac{nisq} devices, either the precision
in which quantum computers can be manipulated has to improve or error-correcting
codes have to be implemented~\cite{shor_quantum_nodate}. Currently both techniques are
being heavily researched and in conjunction will lead to the next generation of
quantum devices, namely fault-tolerant quantum computers. \

A technology that right now has gained a bigger presence in our society is \ac{ml}.
There have been many important breakthroughs for \ac{ml} in the past few years, with
uses in natural language processing, computer vision, anomaly detection, and many
more fields~\cite{bommasani_opportunities_2022}. Nowadays \ac{ml} has a big impact
in society, and even though it depicts big opportunities for improvement in
society it also represents significant risks. \

Quantum computing and \ac{ml} are information processing techniques that have
improved significantly in recent years. This lead to the natural desire of
harnessing the advantages of both and to the emergence of a new field of study
denominated \ac{qml}~\cite{schuld_machine_2021}. \ac{qml} explores several ideas
like whether quantum devices are better at \ac{ml} than classical computers or
if quantum information adds new data that affects how machines recognize patterns. \

Returning to the possible risks that \ac{ml} might encounter, several attacks have
been developed to force a \ac{ml} model to missclassify an input~\cite{szegedy_intriguing_2014}.
These attacks are denominated adversarial attacks and are based on crafting specific
input data that has been slightly modified to cause the model to erroneously classify
the input. These small modifications are imperceptible for humans. At the beginning,
when the first adversarial attacks were developed, they needed to know the
architecture of the model to be able to fool it. Nevertheless, it was proved
that adversarial attacks are transferable between models with the same use
case, without knowing the architecture of the model or the dataset it was
trained on~\cite{papernot_transferability_2016}. \

Adversarial training was developed in order to defend \ac{ml} models against
adversarial attacks~\cite{goodfellow_explaining_2015, szegedy_intriguing_2014}.
Adversarial training consists of including adversarial samples into the training
of the \ac{ml} model to better generalize its classification. This mechanism
has a tradeoff, in which the accuracy of the model lowers, while increasing the
resilience to adversarial attacks~\cite{kurakin_adversarial_2017}. \

In classical \ac{ml}, noise in training has been shown to improve generalization
performance and local optima avoidance~\cite{ciliberto_quantum_2018}. This
property from noise is particularly interesting in \ac{nisq} devices, as their
inherent noise might be able to improve \ac{qml} performance, accuracy and
resilience against adversarial attacks. \

\section{Research Goals}\label{section:research_goals} \

The main goal of this thesis is to investigate the effects of quantum noise
on the resilience of \ac{qml} models against adversarial
attacks. To achieve this we set four intermediate goals that build
upon each other to fulfill the main goal. These goals are: \

\begin{enumerate}
    \item \textbf{Train noiseless \ac{qml} models on different datasets:}
            We choose different datasets to train several \ac{qml} models.
            Selecting distinct datasets, each with increasing amount of
            features, will test the baseline performance of the \ac{qml}
            models and set a benchmark for the upcoming noisy models. \
    \item \textbf{Train noisy \ac{qml} models on different datasets:} Once
            the baseline \ac{qml} model accuracy for the different datasets
            has been established, we then train more \ac{qml} models that
            suffer from quantum noise. We choose six different types of
            noise models and for each type we modify the magnitude of the
            noise with five different values. This allows us not only
            to compare noiseless and noisy \ac{qml} models performance,
            but also to observe the impact of quantum noise on the models'
            accuracy. \
    \item \textbf{Perform adversarial attacks on noiseless \ac{qml} models:}
            We perform two different adversarial attacks on the noiseless
            \ac{qml} models for the different datasets. This will allow
            us to obtain the adversarial examples to further evaluate the
            previously trained noisy models. Two different adversarial
            techniques with five different attack strengths are utilized
            to validate the possible effects of quantum noise as a defense
            mechanism against adversarial attacks. \
    \item \textbf{Evaluate noisy \ac{qml} models with adversarial samples:}
            Finally, with the adversarial examples obtained from the noiseless
            models, we test all the noisy \ac{qml} models with the different
            constellations of noise magnitude. We cover different types of noise
            with different noise magnitudes as well as different attack techniques
            with increasing attack strength. This will enable us to verify
            any possible effect caused by quantum noise on \ac{qml} model robustness
            against adversarial attacks. \
\end{enumerate} \

\section{Outline}\label{section:outline} \

This thesis is structured as follows. In Chapter~\ref{chapter:background}
we will introduce the required quantum computing, quantum noise, \ac{qml},
and \ac{aml} concepts important for this thesis. Furthermore, in Chapter
~\ref{chapter:design} we will present the design decisions (and their
justification) taken to implement the experiments aimed at investigating
the research questions. Moreover, in Chapter~\ref{chapter:implementation}
we describe the procedure that the tests follow to provide information
about the research goals. Additionally, in Chapter~\ref{chapter:results}
we provide the results of the conducted experiments and discuss the observed
findings. Finally, a conclusion of the thesis and some suggestions on what
can be further researched are given in Chapter~\ref{chapter:conclusion} and
Chapter~\ref{chapter:future_work} respectively.\